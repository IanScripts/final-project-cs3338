\documentclass[12pt]{article}
\usepackage{geometry}
\usepackage{graphicx}
\usepackage{hyperref}
\usepackage{longtable}
\usepackage{array}
\geometry{margin=1in}

\title{Software Design Document (SDD) \\ \large Amber Molecular Dynamics Web Interface}
\author{Group number 7}
\date{\today}

\begin{document}

\maketitle

\tableofcontents
\newpage

%---------------------------------------
\section*{Version Description}
\addcontentsline{toc}{section}{Version Description}

\begin{longtable}{|m{2.5cm}|m{9cm}|m{3cm}|}
\hline
\textbf{Version} & \textbf{Description} & \textbf{Date Added}\\\hline
1.0 & SDD for Snapshot 1 that includes architecture, UI overview, and database design. & \today\\\hline
\end{longtable}

%---------------------------------------
\section{Introduction}
\subsection{Purpose}
The purpose of this Software Design Document is to describe the
architecture and internal design of the \textbf{Amber Molecular Dynamics Web Interface}.This system provides a user-friendly way to configure,
submit, and monitor molecular dynamics simulations using the
Amber software suite.

\subsection{Intended Audience}
This document is intended for:
\begin{itemize}
    \item Developers implementing and maintaining the system.
    \item Testers validating the system behavior.
    \item Future contributors may want to work on the system in the future
\end{itemize}

\subsection{System Overview}
The system exposes a web-based interface for running Amber MD
simulations. Users can:
\begin{itemize}
    \item Upload or define input structures and parameter files.
    \item Configure simulation parameters, things like force fields, temperature, time steps, etc.
    \item Submit jobs to a backend Amber installation or job queue.
    \item View job status and retrieve basic output information.
\end{itemize}

%---------------------------------------
\section{System Architecture}
\subsection{High-Level Architecture}
At a high level, the system follows a multi-tier architecture:
\begin{itemize}
    \item \textbf{Client-side (Frontend)}: Web UI where users configure simulations,
          view job status, and download results.
    \item \textbf{Server-side (Backend)}: REST endpoints or controllers that
          validate requests, persist data, and interact with Amber.
    \item \textbf{Database}: Stores user accounts, simulation configurations,
          job metadata, and links to result files.
\end{itemize}

\subsection{System Workflow}
A typical workflow:
\begin{enumerate}
    \item User logs in to the web interface.
    \item User creates a new simulation configuration and uploads input files.
    \item The backend validates the configuration and saves it in the database.
    \item The job execution layer generates Amber input scripts and submits a job.
    \item Amber runs the molecular dynamics simulation on the server or cluster.
    \item The system periodically checks job status and updates the user interface.
    \item When complete, the user can download output files or view summary data.
\end{enumerate}

\subsection{Component Breakdown}
\subsubsection{Server-Side Components}
\begin{itemize}
    \item \textbf{Authentication Controller}: Handles login, registration, and session management.
    \item \textbf{Simulation Controller}: CRUD operations for simulation setups.
    \item \textbf{Job Controller}: Submits and monitors Amber jobs.
    \item \textbf{Result Controller}: Exposes endpoints to retrieve results.
\end{itemize}

\subsubsection{Client-Side Components}
\begin{itemize}
    \item \textbf{Login / Registration Pages}
    \item \textbf{Dashboard Page}: List of user simulations and job statuses.
    \item \textbf{Simulation Form Page}: UI to select force field, temperature, time, etc.
    \item \textbf{Job Detail Page}: Detailed status and links to outputs.
\end{itemize}

\subsubsection{External Services / Integrations}
\begin{itemize}
    \item \textbf{Amber MD Package}: Can be used via command-line tools installed on the server.
    \item \textbf{Filesystem or Storage}: Stores input and output files.
\end{itemize}

%---------------------------------------
\section{User Interface}
\subsection{How to Use the System}
\begin{enumerate}
    \item Navigate to the home page and log in or create an account.
    \item From the dashboard, click ``New Simulation".
    \item Upload the structure file and any required parameter files.
    \item Fill out simulation settings
    \item Submit the job and monitor status on the dashboard.
    \item After completion, click on the job to view details and download outputs.
\end{enumerate}

%---------------------------------------
\section{Database Design}
The database stores:
\begin{itemize}
    \item \textbf{Users}: login credentials and profile information.
    \item \textbf{Simulations}: configuration parameters and metadata.
    \item \textbf{Jobs}: job status, timestamps, and Amber command used.
    \item \textbf{Files}: references to input and output files on disk.
\end{itemize}
%---------------------------------------
\section{Glossary}
\begin{longtable}{|m{4cm}|m{10cm}|}
\hline
\textbf{Acronym} & \textbf{Definition} \\ \hline
MD & Molecular Dynamics \\ \hline
UI & User Interface \\ \hline
API & Application Programming Interface \\ \hline
DB & Database \\ \hline
PDB & Protein Data Bank file format \\ \hline
\end{longtable}

%---------------------------------------
\section{References}
\begin{itemize}
    \item Amber Molecular Dynamics Official Website: \url{https://ambermd.org}
    \item R. Salomon-Ferrer, D.A. Case, R.C. Walker. An overview of the Amber biomolecular simulation package.
    \item Course slides and assignment description for CS 3338.
\end{itemize}

\end{document}
