\documentclass[12pt]{article}
\usepackage{geometry}
\usepackage{graphicx}
\usepackage{hyperref}
\usepackage{longtable}
\usepackage{array}
\geometry{margin=1in}

\title{Software Design Document (SDD) \\ \large Amber Molecular Dynamics Web Interface}
\author{Group number 7}
\date{\today}

\begin{document}

\maketitle

\tableofcontents
\newpage

%---------------------------------------
\section*{Version Description}
\addcontentsline{toc}{section}{Version Description}

\begin{longtable}{|m{2.5cm}|m{9cm}|m{3cm}|}
\hline
\textbf{Version} & \textbf{Description} & \textbf{Date Added} \\ \hline

1.0 & Initial SDD including architecture overview, UI overview, and workflow. & \today \\ \hline

1.1 & Added Component Breakdown section (Authentication, Simulation, Job, Result Controllers). & \today \\ \hline

1.2 & Added Database Design section (Users, Simulations, Jobs, and Files tables). & \today \\ \hline
\begin{figure}
    \centering
    \includegraphics[width=1.0\linewidth]{architecture_diagram.png}
    \caption{Enter Caption}
    \label{fig:placeholder}
\end{figure}
\end{longtable}


%---------------------------------------
\section{Introduction}
\subsection{Purpose}
The purpose of this Software Design Document is to describe the
architecture and internal design of the \textbf{Amber Molecular Dynamics Web Interface}.
This system provides a user-friendly way to configure, submit, and monitor MD simulations
using the Amber software suite.

\subsection{Intended Audience}
This document is intended for:
\begin{itemize}
    \item Developers implementing and maintaining the system.
    \item Testers validating system behavior.
    \item Future contributors to the application.
\end{itemize}

\subsection{System Overview}
Users can:
\begin{itemize}
    \item Upload input structures and parameter files.
    \item Configure simulation parameters such as force fields, temperature, time steps.
    \item Submit jobs to a backend Amber installation.
    \item View job status and retrieve output files.
\end{itemize}

%---------------------------------------
\section{Site Breakdown}
\begin{itemize}
    \item \textbf{Home Page} – Entry point with login/register buttons.
    \item \textbf{Dashboard Page} – Displays all user simulations and job statuses.
    \item \textbf{Simulation Setup Page} – Form for simulation parameters and file uploads.
    \item \textbf{Job Detail Page} – Shows job information, timestamps, and logs.
    \item \textbf{Results Download Page} – Provides links to simulation output files.
    \item \textbf{Settings Page} (optional future work) – Manage compute settings and user preferences.
\end{itemize}

%---------------------------------------
\section{System Architecture}

\subsection{High-Level Architecture}
At a high level, the system follows a multi-tier architecture:
\begin{itemize}
    \item \textbf{Frontend}: Web UI for simulation configuration and monitoring.
    \item \textbf{Backend}: REST controllers that validate requests, manage jobs, and interact with Amber.
    \item \textbf{Database}: Stores users, simulations, job metadata, and file references.
\end{itemize}

\begin{figure}[h]
\centering
\includegraphics[width=\textwidth]{architecture_diagram.png}
\caption{High-Level Architecture Diagram of the Amber MD Web Interface}
\end{figure}

\subsection{System Workflow}
\begin{enumerate}
    \item User logs in to the web interface.
    \item User creates a simulation configuration and uploads files.
    \item Backend validates inputs and saves configuration.
    \item Job execution layer generates Amber input scripts and runs the job.
    \item Amber runs the MD simulation on server/cluster.
    \item Status is periodically checked and UI is updated.
    \item User downloads results after completion.
\end{enumerate}

\subsection{Component Breakdown}
\subsubsection{Server-Side Components}
\begin{itemize}
    \item \textbf{Authentication Controller}
    \item \textbf{Simulation Controller}
    \item \textbf{Job Controller}
    \item \textbf{Result Controller}
\end{itemize}

\subsubsection{Client-Side Components}
\begin{itemize}
    \item Login / Registration Pages
    \item Dashboard
    \item Simulation Form Page
    \item Job Detail Page
\end{itemize}

\subsubsection{External Services}
\begin{itemize}
    \item Amber MD package commands.
    \item Filesystem for storage.
\end{itemize}

%---------------------------------------
\section{User Interface}

\subsection{UI Overview}
The interface presents a guided workflow for running simulations.

\subsection{Page Breakdown}

\subsubsection{Dashboard Page}
Displays all user simulations and their statuses (Pending, Running, Completed, Failed)
along with buttons for viewing details and downloading outputs.

\subsubsection{Simulation Setup Page}
Wizard-like interface where the user uploads structure files,
selects force fields, sets temperature, and other MD parameters.

\subsubsection{Job Detail Page}
Displays job logs, timestamps, run parameters, Amber command used,
and progress indicators.

\subsubsection{Results Download Page}
Provides organized download links for:
\begin{itemize}
    \item Trajectory files
    \item Topology files
    \item Energy logs
    \item Restart files
\end{itemize}

%---------------------------------------
\section{Database Design}
The database stores:
\begin{itemize}
    \item \textbf{Users}: login credentials, metadata.
    \item \textbf{Simulations}: configuration parameters and metadata.
    \item \textbf{Jobs}: statuses, timestamps, and commands.
    \item \textbf{Files}: references to input/output paths.
\end{itemize}

%---------------------------------------
\section{Glossary}
\begin{longtable}{|m{4cm}|m{10cm}|}
\hline
\textbf{Acronym} & \textbf{Definition} \\ \hline
MD & Molecular Dynamics \\ \hline
UI & User Interface \\ \hline
API & Application Programming Interface \\ \hline
DB & Database \\ \hline
PDB & Protein Data Bank \\ \hline
\end{longtable}

%---------------------------------------
\section{References}
\begin{itemize}
    \item Amber Official Website: \url{https://ambermd.org}
    \item Course materials from CS 3338.
    \item Amber Official GitHub
\begin{figure}
        \centering
    \end{figure}
        \url{https://github.com/amber-md}
\end{itemize}

\end{document}
